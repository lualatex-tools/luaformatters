\documentclass{article}

\usepackage[color]{luatemplates}
\setupLuaTemplates{mwe}{luatemplates-config-mwe.lua}

\title{\package{luatemplates} -- MWE}

\begin{document}
\maketitle

This is a \textsc{mwe} for setting up a document with \package{luatemplates}
support.  The commands \cmd{package} and \cmd{cmd} have been set up using
\emph{style} templates, while \cmd{XXX}, which “X-es out” a given string
(\cmd{XXX\{foo-bar\}} => \XXX{foo-bar}) is defined as a simple formatter
function.  Note how \emph{all} the style programming is done inside the Lua
table (\texttt{luatemplates-config-mwe.lua}) while the interfacing between
\LaTeX\ and Lua is done with two lines in the document preamble.

\end{document}
