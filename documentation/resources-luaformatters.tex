
% This is the manual for the luaformatters package.
% Along with the files
% - luaformatters-manual-templates.lua
% - examples/*.lua and examples/*.tex
% - luaformatters-manual-config-mwe.lua
% the source itself serves as documentation to the package,
% its use and programming.
%%%%%%%%%%%%%%%%%%%%%%%%%%%%%%%%%%%%%%%%%%%%%%%%%%%%%%%%%%%

\documentclass[12pt]{scrartcl}
\usepackage{luaformattersmanual}

% Use or require the `luaformatters` package with (some of the) options
% NOTE: option self-documentation implicitly loads minted, so that doesn't
% have to be included explicitly for the writing of this document.
\usepackage[
color,              % use colors
self-documentation, % Create the self-documentation commands
]{luaformatters}

% Set up the Formatters object, passing a configuration file name.
% The Formatters object will later be globally available to Lua by the name
% manual_templates and could be adressed as such from any \directlua command.
% luaformatters-manual-config.lua must be findable by LaTeX.
\addLuaFormatters{luaformatters-manual}

% The formatters are organized in a modular fashion, distributed over several
% declaration files.  The purpose in *this document* is to be able to list
% the following files separately at the end of the manual, but it also shows
% an approach that can be used for other purposes of modular organization.
\addLuaFormatters{examples/luaformatters-closure}
\addLuaFormatters{examples/luaformatters-format}
\addLuaFormatters{examples/luaformatters-local-formatters.lua}
\addLuaFormatters{examples/luaformatters-additionals}


\title{\luaformatters\ -- Manual Materials}
\subtitle{v0.8}
\author{Urs Liska}
\date{\HEADdate}

\begin{document}

\maketitle

\section{Stuff for possible reuse}

\subsubsection{Creating Macros}
\label{sec:intro-creating-macros}

\section{Stuff likely to be thrown out}

\subsection{A First Glimpse / The Big Idea}
\label{sec:a-first-glimpse}

When writing packages specifically for Lua\LaTeX, one typically wants to work as
much as possible in the Lua domain, doing things in plain \LaTeX\ only when
necessary.  One common idiom in this context is to get the data from the \TeX\
to the Lua domain as quickly as possible, process it there and only write the
final result back to the \TeX\ document.  However, this can involve significant
overhead that is not always worth the effort and can leave the code in a
confusing mix of \LaTeX\ and Lua.  A useful approach is to prepare all coding in
a Lua module and provide \LaTeX\ macros that basically serve as mere
\emph{interfaces} to that infrastructure.  The basic idea of this package is to
make this process as automatic and painless as possible by providing tools to
\emph{declare} \term{Formatters} (string templates and formatter functions) in
one Lua table and have them exposed as \LaTeX\ macros mostly automatically.  The
following excerpt from a Lua configuration file%
\footnote{Throughout this manual examples will use the variable \luavar{MANUAL},
	which is the actual variable used in the configuration file for this manual,
	which is also listed in \vref{sec:examples:manual-templates}.}% :

\begin{minted}{lua}
MANUAL:add_formatter('cmd', [[\textbf{\textbackslash <<<name>>>}]])
\end{minted}

\noindent will automatically create a command \luaMacroDocInline{cmd} with one
mandatory argument that can be used in the \LaTeX\ document like this:
\luaMacroDocInline[demo,args=usepackage]{cmd} (and is so throughout this
document) without any further set-up work.  And a Lua function

\label{code:function}
\begin{minted}{lua}
function MANUAL.formatters:reverse(text, options)
local result = text:reverse()
if options.upper then
result = result:upper()
end
return result
end
\end{minted}

\noindent will create a macro \luaMacroDocInline{reverse} with an optional
argument that can be used like

\luaMacroDoc[demo,nocomment,%
demosep=\par\noindent producing ,%
args={upper,The brown fox}]{reverse}%
in the output.  There are four things to highlight at this point, which are
detailed further in the following sections of the manual:

\begin{itemize*}
	\item The handling of optional and mandatory arguments for the \LaTeX\ macro is
	done automatically, inferring the details directly from the Lua function's signature, and also from string templates when possible.
	\item Through the use of \package{lyluatexoptions} the handling of optional
	\texttt{key=value} arguments has become incredibly simple.  With some more
	administrative effort (setting up one's own options instance) their keys and
	values can even be validated.
	\item The package provides helper functions and encourages to go forward with
	this approach to follow modular design principles.
	\item Output can be colored automatically and conditionally, without having to
	hard-code coloring into the macros. This can equally be used to actually create
	colored output and to use coloring as a visual checker for correct entry.
\end{itemize*}


\end{document}